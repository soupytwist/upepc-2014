\documentclass[11pt]{article}
\usepackage{tabularx}
\usepackage{amsmath}
\usepackage[margin=1in]{geometry}

%Gummi|065|=)
\title{\textbf{Picture Day}}
\date{}
\begin{document}

\maketitle

\section{Problem Statement}
On group picture day, the entire team gathers outside to take a series of
pictures for the company calendar. This year, human resources has decided
that the employees will stand in the same order for every picture, but they
may be asked to wear different color company shirts in each shots.
\\\\
For example, Alice, Bob and Charlie may wear red, yellow, and green shirts
(respectively) in one picture and red, blue, and red shirts in another picture.
There will be a number of pictures taken, each of which has its own arrangement
of colors. The pictures, of course, can be taken in any order.
\\\\
To prevent this task from taking all day, you have been asked to plan the shots
such that the employees will need to change their shirt the least number of times.
\\\\
\emph{Note}: For this problem, you are only asked to output the minimum number of changes,
not the ordering. In addition, you are to assume that no employees are wearing company
shirts at the start, and it does not matter what shirt each employee ends up with.

\section{Input}
The first line of input contains a single integer $\boldsymbol{P}$, $(\boldsymbol{1} \le \boldsymbol{P} \le \boldsymbol{100})$, which is the number of data sets that follow. Each
data set should be processed identically and independently.
\\\\
Each data set begins with a single line that contains $\boldsymbol{K}$, the data
set number, followed by $\boldsymbol{S}$, $(\boldsymbol{2} \le \boldsymbol{S} \le \boldsymbol{12})$
which is the number of of photos that are to be taken and  $\boldsymbol{E}$,
$(\boldsymbol{1} \le \boldsymbol{E} \le \boldsymbol{100})$, the number of employees
that will be a part of the pictures. The next $\boldsymbol{P}$ lines contain $\boldsymbol{E}$
characters each, which indicate the shirt color that each employee will wear for a
particular photo. The shirt colors are described by a single letter \texttt{a}-\texttt{z},
where e.g ``\texttt{a}'' represents some color.

\section{Output}
For each data set there is a single line of output. The single line of output
consists of the data set number $\boldsymbol{K}$, followed by a single space
followed by the minimum number of shirt changes that are necessary to take all
the photos.

\section{Test Data}
\begin{tabularx}{\textwidth}{|X|X|}
	\hline
	Input & Output \\ \hline
	\parbox[t]{5cm}{
	\texttt{3\\
			1 2 9\\
			aaabbbccc\\
			bbbaaaccc\\
			2 3 4\\
			aabb\\
			babb\\
			aaab\\
			3 4 5\\
			aabbc\\
			aaaab\\
			abbcc\\
			aaabb\\
	}} &
	\parbox[t]{5cm}{
	\texttt{1 6\\
			2 2\\
			3 5\\
			}}\\
	\hline
\end{tabularx}
\end{document}