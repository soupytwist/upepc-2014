\documentclass[11pt]{article}
\usepackage{tabularx}
\usepackage{fancybox}
\usepackage{amsmath}
\usepackage[margin=1in]{geometry}
\usepackage{graphicx}

%Gummi|065|=)
\title{\textbf{Scoring Upvotes}}
\author{Nick Smith}
\date{}
\begin{document}

\maketitle

\section{Problem Statement}
Reddit, the community-driven social news and entertainment site, places a
strong emphasis on user-submitted content and comments. Due to the nature
of user-created comments, it is important to have a way to differentiate
the ``good'' comments from the ``bad'' comments. On Reddit, this is
implemented using an upvote/downvote system, where users can judge each
other's comments to pick out which contribute to the conversation and
which are useless.
\\\\
In 2009, Reddit switched their default comment sorting algorithm from ``top''
to ``best'' sorting. The new ``best'' sorting ranks comments based on the \underline{lower bound}
of a comment's Wilson score. The Wilson score is a confidence interval
based on the number of upvotes and downvotes a comment has received which is
used to determine where it belongs in the ranking.
\\\\
The Wilson score is a confidence interval defined by the following formula:
\begin{center}
\fontsize{16pt}{16pt}
$\left(\ \hat{p} - \frac{z^2}{2n} \pm z\sqrt{\left[\hat{p}(1 - \hat{p}) + \frac{z^2}{4n}\right]/n}\ \right)/(1 + \frac{z^2}{n})$
\end{center}
Where $\hat{p}$ is the \emph{observed} fraction of positive ratings, $n$ is
the total number of ratings, and $z$ is the appropriate z-score for the
confidence interval (Reddit uses a 95\% confidence interval, so for this you
can assume that $z = 1.96$).
\\\\
Reddit user \texttt{/u/Here\_Comes\_the\_King} posts a comment on a popular
thread in \texttt{/r/trees}. Given the current upvote/downvote counts of
\texttt{/u/Here\_Comes\_the\_King}'s comment and the top comment, and the
rate at which each comment is being upvoted,
compute how long will it take \texttt{/u/Here\_Comes\_the\_King}'s comment
to overtake the current highest-ranked comment.

\section{Input}
The first line of input contains a single integer $\boldsymbol{P}$,
$(\boldsymbol{1} \le \boldsymbol{P} \le \boldsymbol{1000})$, which is the
number of data sets that follow. Each data set should be processed identically
and independently.
\\\\
Each data set begins with a single line that contains $\boldsymbol{K}$, the data
set number, followed by the integer numbers $\boldsymbol{U_t}\text,\ \boldsymbol{D_t},\ 
\boldsymbol{U_k}\text,\ \boldsymbol{D_k}$, $(\boldsymbol{1} \le \boldsymbol{U_t},
\boldsymbol{U_k} \le \boldsymbol{500},\ \boldsymbol{0} \le \boldsymbol{D_t},
\boldsymbol{D_k} \le \boldsymbol{500})$ respectively, the number of upvotes and
downvotes for the top comment and the number of upvotes and downvotes for
\texttt{/u/Here\_Comes\_the\_King}'s comment. These are followed by two more
integers $\boldsymbol{R_t}\text{ and }\boldsymbol{R_k}$, $(\boldsymbol{1} \le \boldsymbol{R_t},
\boldsymbol{R_k} \le \boldsymbol{50})$ which indicate the number
of upvotes that each comment is receiving \underline{per second}. These rates
will never change and neither comment will receive any additional downvotes.

\section{Output}
For each data set there is a single line of output. The single line of output
consists of the data set number $\boldsymbol{K}$, followed by a single space
followed by the number of seconds (do not consider fractions of seconds) it
will take until the top comment is overtaken. We will consider the top comment
to be overtaken when its comment score is less than or equal to the user's
comment's score. It is also possible that the top comment is never overtaken,
in which case your program should output the text ``\texttt{NEVER}''
(without quotes) instead.

\section{Test Data}
\begin{tabularx}{\textwidth}{|X|X|}
	\hline
	Input & Output \\ \hline
	\parbox[t]{5cm}{
	\texttt{3\\
			1 100 0 50 0 0 1\\
			2 80 10 40 1 1 2\\
			3 100 20 50 10 1 1\\
	}} &
	\parbox[t]{5cm}{
	\texttt{1 50\\
			2 ??\\
			3 NEVER\\
	}}\\
	\hline
\end{tabularx}

\end{document}



































