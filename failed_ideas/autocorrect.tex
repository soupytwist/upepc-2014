\documentclass[11pt]{article}
\usepackage{tabularx}
\usepackage{fancybox}
\usepackage{amsmath}
\usepackage[margin=1in]{geometry}

%Gummi|065|=)
\title{\textbf{Shifty Auto-Correct}}
\date{}
\begin{document}

\maketitle

\section{Problem Statement}
The latest version of the Blandroid mobile OS has a new auto-correct
option that has been causing trouble for some users. The feature allows
the user to enter rules into their keyboard software so that when a particular
word is matched, it is replaced with another word.
\\\\
Given the following rules:
\begin{align}
\texttt{teh} &\rightarrow \texttt{the} \\
\texttt{youre} &\rightarrow \texttt{you're}
\end{align}
When the user types the phrase ``\texttt{youre teh best!}'', this will be
automatically replaced with the intended ``\texttt{you're the best!}''. The
problem is that this feature was not tested thoroughly and it has been shown
to replace partial words. For example, with the above rules, if the user
types ``\texttt{tehran}'', it will be replaced by ``\texttt{theran}''. In
addition, the developer decided to apply the auto-correct replacement to the
whole text every time it is changed. Rules are always applied in the order they
are given, and when a replacement is made, the algorithm restarts at the first rule.
\\\\
Adding the rule:
\begin{align}
\texttt{re} &\rightarrow \texttt{are}
\end{align}
Causes the replacement:
\begin{enumerate}
\item ``\texttt{youre teh best!}''
\item ``\texttt{youre the best!}''
\item ``\texttt{you're the best!}''
\item ``\texttt{you'are the best!}''
\item ``\texttt{you'aare the best!}''
\item ``\texttt{you'aaare the best!}''
\item ``\texttt{you'aaaare the best!}'' \ldots continues indefinitely
\end{enumerate}
\bigskip
You have been commissioned to write a program to detect when the replacement
rules cause the algorithm to enter into an infinite loop, as shown in the
example above.

\section{Input}
The first line of input contains a single integer $\boldsymbol{P}$,
$(\boldsymbol{1} \le \boldsymbol{P} \le \boldsymbol{1000})$, which is the
number of data sets that follow. Each data set should be processed identically
and independently.
\\\\
Each data set begins with a single line that contains $\boldsymbol{K}$, the data
set number, followed by $\boldsymbol{R}$,
$(\boldsymbol{1} \le \boldsymbol{R} \le \boldsymbol{1000})$ which is the number
of replacement rules that will follow.
\\\\
The next $\boldsymbol{R}$ lines contain two strings $(\boldsymbol{w_\text{old},\ w_\text{new}})$
which, respectively, indicate a word that should be replaced and what word to replace it with.
Each word is composed of between 1 and 10 non-whitespace characters and they are separated
by a single space. Replacements are always case sensitive, i.e ``\texttt{hello}'' is treated
differently than ``\texttt{Hello}''.

\section{Output}
For each data set there is a single line of output. The single line of output
consists of the data set number $\boldsymbol{K}$, followed by a single space
followed by the letter \texttt{Y} if the rules create an infinite loop or the
letter \texttt{N} if they do not.

\section{Test Data}
\begin{tabularx}{\textwidth}{|X|X|}
	\hline
	Input & Output \\ \hline
	\parbox[t]{5cm}{
	\texttt{3\\
			1 3\\
			youre you're\\
			teh the\\
			re are\\
			2 2\\
			hell heck\\
			helo hello\\
			3 3\\
			they're there\\
			their they're\\
			there their\\
	}} &
	\parbox[t]{5cm}{
	\texttt{1 Y\\
			2 N\\
			3 Y\\
	}}\\
	\hline
\end{tabularx}

\end{document}