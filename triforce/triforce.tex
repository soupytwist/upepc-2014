\documentclass[11pt]{article}
\usepackage{tabularx}
\usepackage{fancybox}
\usepackage{amsmath}
\usepackage[margin=1in]{geometry}
\usepackage{tikz}
\usepackage{wrapfig}

%Gummi|065|=)
\title{\textbf{Triforce}}
\date{}
\begin{document}

\maketitle

\section{Problem Statement}

\definecolor{ltgray}{gray}{0.92}
\begin{wrapfigure}{r}{3cm}
\begin{tikzpicture}[scale=0.25]
\coordinate (A) at (3, 5.196);
\coordinate (B) at (0, 0);
\coordinate (C) at (6,0);
\coordinate (D) at (0, 0);
\coordinate (E) at (-3, -5.196);
\coordinate (F) at (3, -5.196);
\coordinate (G) at (6, 0);
\coordinate (H) at (3, -5.196);
\coordinate (I) at (9, -5.196);

\draw[fill=ltgray] (A) -- (B) -- (C) -- cycle;
\draw[fill=ltgray] (D) -- (E) -- (F) -- cycle;
\draw[fill=ltgray] (G) -- (H) -- (I) -- cycle;
\end{tikzpicture}
\end{wrapfigure}
A ``Triforce'', the fictional relic from Nintendo's \emph{The Legend of Zelda}
video games, is composed of three identical equilateral triangles that connect
at their vertices to form a larger triangle (as pictured).
\\\\
Given a set of three triangles, defined by their vertices on a 2D plane,
your task is to determine whether they form a Triforce. The shape that is
formed may be of arbitrary size and can be rotated at any angle.\\

\section{Input}
The first line of input contains a single integer $\boldsymbol{P}$,
$(\boldsymbol{1} \le \boldsymbol{P} \le \boldsymbol{1000})$, which is the
number of data sets that follow. Each data set should be processed identically
and independently.
\\\\
Each data set begins with a single line that contains $\boldsymbol{K}$, the data
set number, followed by nine (9) lines, each of which contains a pair of decimal numbers
$(\boldsymbol{X},\ \boldsymbol{Y})$, $(-5000 \le \boldsymbol{X},\boldsymbol{Y} \le 5000)$
which represent coordinates on a 10,000 $\times$ 10,000 plane. The first three
coordinates make up one triangle, the middle three the second triangle, and
the last three the third triangle.

\section{Output}
For each data set there is a single line of output. The single line of output
consists of the data set number $\boldsymbol{K}$, followed by a single space
followed by the letter \texttt{Y} if triangles form a Triforce or the letter
\texttt{N} if they do not.

\section{Test Data}
\begin{tabularx}{\textwidth}{|X|X|}
	\hline
	Input & Output \\ \hline
	\parbox[t]{5cm}{
	\texttt{3\\
			1\\
      TODO
	}} &
	\parbox[t]{5cm}{
	\texttt{1 Y\\
			2 Y\\
			3 N\\
	}}\\
	\hline
\end{tabularx}

\end{document}



































