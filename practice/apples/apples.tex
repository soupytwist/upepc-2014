\documentclass[a4paper,11pt,oneside]{article}
\usepackage{tabularx}
\usepackage{fancybox}
\usepackage{amsmath}
\usepackage[top=1.5in,bottom=1in,right=1in,left=1in,headheight=65pt]{geometry}
\usepackage{graphicx}
\usepackage{fancyhdr}

%Gummi|065|=)
\begin{document}
\pagestyle{fancy}

\fancyhead[L]{Stevens UPEPC\\Practice Problem}
\fancyhead[C]{{\LARGE\textbf{Apples}}}
\fancyhead[R]{\includegraphics[height=32pt]{../../stevenslogo.eps}\ \ \ \includegraphics[height=32pt]{../../upelogo.png}}


\section{Problem Statement}
Johnny has $A$ apples. Bob gives him $B$ more. Johnny then gives half, rounded up, to Sally.\\
How many apples does Johnny have left?

\section{Input}
The first line of input contains a single integer $\boldsymbol{P}$,
$(\boldsymbol{1} \le \boldsymbol{P} \le \boldsymbol{1000})$, which is the
number of data sets that follow. Each data set should be processed identically
and independently.
\\\\
Each data set consists of a single line that contains $\boldsymbol{K}$, the data
set number, followed by two numbers $\boldsymbol{A}$ and $\boldsymbol{B}$,
$(\boldsymbol{0} \le \boldsymbol{A}, \boldsymbol{B} \le 5 \times 10^{18})$, where
$\boldsymbol{A}$ is the number of apples that Johnny starts with and $\boldsymbol{B}$ is
the number of apples that Bob gives him.


\section{Output}
For each data set there is a single line of output consisting of the data set number $\boldsymbol{K}$,
followed by a space followed by the number of apples that Johnny has left at the end of the exchanges.

\section{Test Data}
\begin{tabularx}{\textwidth}{|X|X|}
	\hline
	Input & Output \\ \hline
	\parbox[t]{5cm}{
	\texttt{3\\
			1 4 2\\
			2 1 2\\
			3 7 7\\
	}} & \parbox[t]{5cm}{
	\texttt{1 3\\
			2 1\\
			3 7\\
	}}\\
	\hline
\end{tabularx}
\\\\
\\
\textbf{Test Case \#1}\\
Johnny starts with 4 apples and gets 2 more. He gives half of his 6 apples, 3, to Sally,
leaving him with 3.
\\\\
\textbf{Test Case \#2}\\
Johnny starts with 1 apple and gets 2 more. He gives 2 apples to Sally, being as he is
generous and rounds up, leaving him with only 1.
\\\\
\textbf{Test Case \#3}\\
Johnny starts with 7 apples, gets 7 more, and gives 7 away, leaving him with 7.

\end{document}