\documentclass[11pt]{article}
\usepackage{tabularx}
\usepackage{amsmath}
\usepackage[top=1.5in,bottom=1in,right=1in,left=1in,headheight=65pt]{geometry}
\usepackage{graphicx}
\usepackage{fancyhdr}

%Gummi|065|=)
\title{\textbf{IPvX}}
\date{}
\begin{document}
\pagestyle{fancy}

\fancyhead[L]{Stevens UPEPC\\12 April 2014}
\fancyhead[C]{{\LARGE\textbf{IPvX}}}
\fancyhead[R]{\includegraphics[height=32pt]{../stevenslogo.eps}\ \ \ \includegraphics[height=32pt]{../upelogo.png}}

\section{Problem Statement}

\subsection{Background}
IPv4 addresses are 32 bits long, typically represented by 4 integer numbers ranging 0-255, separated by periods. For instance, \texttt{155.246.89.22} is the human-readable representation of an IP address, in binary this is \texttt{10011011 11110110 01011001 00010110}. To create networks, IPv4 addresses are divided into two parts: the \emph{network prefix} and the \emph{host component}. Two IP addresses are on the same network if they have the same \emph{network prefix}.
\\\\
The \emph{netmask} defines the bits that will be used to determine the network prefix portion of an IP address. In IPv4 this is represented by 32 bits where a \texttt{1} indicates that the bit is part of the network prefix and a \texttt{0} indicates it is part of the host component.
\\\\
If the network prefix is 16 bits long, then the above IP address \texttt{155.246.89.22/16} is on the network \texttt{10011011 11110110 00000000 00000000}. This is found simply by performing a bitwise \texttt{AND} operation on the IP address and the netmask. The netmask, in this case, is \texttt{255.255.0.0} or 
\texttt{11111111 11111111 00000000 00000000} (16 leading \texttt{1}'s, hence the 16-bit network prefix).

\subsection{Problem}
The proposed IPvX specification takes a different approach for addressing. Whereas IPv4 may use the first (most significant) 16 bits of a 32-bit address for the network prefix, IPvX allows for addresses of length $m$, where $1 \le m \le 64$, which can use up to $m-1$ bits for the network component. Additionally, the $n$ bits used for the network component (bits that are set to \texttt{1}) do not have to be the first $n$ bits in the string, however the bits for the host component (indicated by \texttt{0}'s) must be contiguous.
\\\\ 
\begin{tabularx}{\textwidth}{|X|X|}
	\hline
	\texttt{11111111 00000000 = 65,280} & valid (m = 16, n = 8)\\ \hline
	\texttt{11111111 11111100 = 65,532} & valid (m = 16, n = 14)\\ \hline
	\texttt{11110000 00001111 = 61,455} & valid (m = 16, n = 8)\\ \hline
	\texttt{11110000 11110000 = 61,680} & invalid (m = 16, n = 8)\\ \hline
\end{tabularx}
\\\\
Given the specification, you must write a program to output all of the valid netmasks for a given IPvX configuration.

\section{Input}
The first line of input contains a single integer $\boldsymbol{P}$, $(\boldsymbol{1} \le \boldsymbol{P} \le \boldsymbol{10,000})$, which is the number of data sets that follow. Each
data set should be processed identically and independently.
\\\\
Each data set begins with a single line that contains $\boldsymbol{K}$, the data
set number, followed by $\boldsymbol{M}$, $(\boldsymbol{1} \le \boldsymbol{M} \le \boldsymbol{64})$
which is the number of bits in each address and  $\boldsymbol{N}$,
$(\boldsymbol{1} \le \boldsymbol{N} < \boldsymbol{M})$, the number of bits that make up the network prefix.

\section{Output}
For each data set there is a single line of output. The single line of output
consists of the data set number $\boldsymbol{K}$, followed by a single space
followed by, \textbf{in increasing order}, a space-separated list of the valid netmasks
in their decimal representation (e.g $\texttt{11110000}_2 = 240_{10}$).
\begin{center}\emph{Note}: It's okay to have leading or trailing spaces; these will simply be ignored.\end{center}

\section{Test Data}
\begin{tabularx}{\textwidth}{|X|X|}
	\hline
	Input & Output \\ \hline
	\parbox[t]{5cm}{
	\texttt{3\\
			1 4 2\\
			2 4 1\\
			3 6 3\\
	}} &
	\parbox[t]{5cm}{
	\texttt{1 3 9 12\\
			2 1 8\\
			3 7 35 49 56\\
			}}\\
	\hline
\end{tabularx}
\\\\\\
\textbf{Test Case \#1}\\
In the first problem, a 2-bit network component is used in 4-bit addresses. Being as the host
component must be contiguous, the valid netmasks are: $\texttt{0011} = 3_{10}$, $\texttt{1001} = 9_{10}$ and $\texttt{1100} = 12_{10}$
\\\\
\textbf{Test Case \#2}\\
In the second problem, a 1-bit network component is used in 4-bit addresses. Valid netmasks are: $\texttt{0001} = 1_{10}$ and $\texttt{1000} = 8_{10}$.
\\\\
\textbf{Test Case \#3}\\
In the third problem, a 3-bit network component is used in 6-bit addresses. Valid netmasks are: $\texttt{000111} = 7_{10}$, $\texttt{100011} = 35_{10}$, $\texttt{110001} = 49_{10}$ and $\texttt{111000} = 56_{10}$.





























\end{document}